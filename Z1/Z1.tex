\documentclass{article}
\usepackage[MeX]{polski} 
\usepackage[utf8]{inputenc} 
\usepackage{graphicx}
\usepackage{wrapfig}
\usepackage{multirow}
\graphicspath{ {./images/} }

\title{Tytuł pracy}
\author{Dominik Belgrau}

\newpage

\begin{document}
\maketitle

\begin{abstract}
    Praca zawiera dzieła literatury takie jak "Treny" J. Kochanowskiego i jakieś śmieszne krzaczki. Źródła na str. \pageref{bb}.
\end{abstract}
\begin{figure}[h!]
\includegraphics[scale=2]{woody.png} \label{o0}
\caption{Woody}
\end{figure}


\newpage

\tableofcontents
\listoftables
\listoffigures

\newpage

\section{Treny}

Źródło - sprawdź str. \pageref{bb}.

\subsection{Tabela} \label{tab}
\begin{table}
\begin{center}
\begin{tabular}{ |c|c|c|c| } 
\hline
Autor & Tren & Ocena \\
\hline
\multirow{3}{8em}{Kochanowski}  & V & 4 \\ 
& VIII & 5 \\ 
& XIX & 6 \\ 
\hline
\end{tabular}
\caption{Treny}
\end{center}
\end{table}

\subsection{Tren VIII}

\textbf{Wielkieś mi uczyniła pustki w domu moim}

Wielkieś mi uczyniła pustki w domu moim,

Moja droga Orszulo, tym zniknieniem swoim!

Pełno nas, a jakoby nikogo nie było:

Jedną maluczką duszą tak wiele ubyło.

Tyś za wszytki mówiła, za wszytki śpiewała,

Wszytkiś w domu kąciki zawżdy pobiegała.

Nie dopuściłaś nigdy matce się frasować

Ani ojcu myśleniem zbytnim głowy psować,

To tego, to owego wdzięcznie obłapiając

I onym swym uciesznym śmiechem zabawiając.

Teraz wszytko umilkło, szczere pustki w domu,

Nie masz zabawki, nie masz rozśmiać się nikomu.

Z każdego kąta żałość człowieka ujmuje,

A serce swej pociechy darmo upatruje.

\subsection{Tren V}

\textit{Jako oliwka mała pod wysokim sadem}

Jako oliwka mała pod wysokim sadem

Idzie z ziemie ku górze macierzyńskim śladem,

Jeszcze ani gałązek, ani listków rodząc,

Sama tylko dopiro szczupłym prątkiem wschodząc:

Tę jesli, ostre ciernie lub rodne pokrzywy

Uprzątając, sadownik podciął ukwapliwy,

Mdleje zaraz, a zbywszy siły przyrodzonej,

Upada przed nogami matki ulubionej -

Takci się mej namilszej Orszuli dostało.

Przed oczyma rodziców swoich rostąc, mało

Od ziemie się co wznióswszy, duchem zaraźliwym

Srogiej Śmierci otchniona, rodzicom troskliwym

U nóg martwa upadła. O zła Persefono,

Mogłażeś tak wielu łzam dać upłynąć płono?

\subsection{Tren XIX}

\texttt{Sen}

Żałość moja długo w noc oczu mi nie dała

Zamknąć i zemdlonego upokoić ciała ;

Ledwie mię na godzinę przed świtaniem swymi

Sen leniwy obłapił skrzydły czarnawymi.

Natenczas mi się matka włamie ukazała,

A na ręku Orszulę moję wdzięczną miała,

Jaka więc po paciorek do mnie przychodziła,

Skoro z swego posłania rano się ruszyła.

Giezłeczko białe na niej, włoski pokręcone,

Twarz rumiana, a oczy ku śmiechu skłonione.

Patrzę, co dalej będzie, aż matka tak rzecze:

"Śpisz, Janie? czy cię żałość twoja zwykła piecze?"

Zatym-em ciężko westchnął i tak mi się zdało,

Żem się ocknął. - A ona, pomilczawszy mało,

Znowu mówić poczęła : "Twój nieutolony

Płacz, synu mój, przywiódł mię w te tu wasze strony

Z krain barzo dalekich, a łzy gorzkie twoje

Przeszły aż i umarłych tajemne pokoje.

Przyniosłam ci na ręku wdzięczną dziewkę twoją,

Abyś ją mógł oglądać jeszcze, a tę swoją

Serdeczną żałość ujął, która tak ujmuje

Sił twoich i tak zdrowie nieznacznie twe psuje,

Jako ogień suchy knot obraca w perzyny,

Darmo nie upuszczając namniejszej godziny.

Czyli nas już umarłe macie za stracone

I którym już na wieki słońce jest zgaszone?

A my, owszem, żywiemy żywot tym ważniejszy,

Czym nad to grube ciało duch jest szlachetniejszy.

Ziemia w ziemię się wraca, a duch, z nieba dany,

Miałby zginąć ani na miejsca swe wezwany?

O to się ty nie frasuj, a wierz niewątpliwie,

Że twoja namilejsza Orszuleczka żywie.

A tu więc takim ci się kształtem ukazała,

Jakoby się śmiertelnym oczom poznać dała.

Ale między anioły i duchy wiecznymi

Jako wdzięczna jutrzenka świeci, a za swymi

Rodzicami się modli, jako to umiała

Z wami będąc. choć jeszcze słów nie domawiała.

Jesliżeć też stąd roście żałość, że jej lata

Pierwej są przyłomione, niżli tego świata

Rozkoszy zażyć mogła? O biedne i płone

Rozkoszy wasze, które tak są usadzone,

Że w nich więcej frasunków i żałości więcej!

Czego ty doznać możesz sam z siebie napręcej!

Ucieszyłeś się kiedy z dziewki swej tak wiele,

Żeby pociecha twoja i ono wesele

Mogło porównać z twoim dzisiejszym kłopotem?

Nie rzeczesz tego, widzę! Także trzymaj o tem,

Jakoś doznał, ani się frasuj, że tak rana

Twojej ze wszech namilszej dziewce śmierć zesłana!

Nie od rozkoszyć poszła; poszłać od trudności,

Od pracej, od frasunków, od łez, od żałośni,

Czego świat ma tak wiele, że by też co było

W tym doczesnym żywocie człowieczeństwu miło,

Musi smak swój utracić prze wielkość przysady,

A przynamniej prze bojaźń nieuchronnej zdrady.

Czegóż płaczesz, prze Boga? Czegóż nie zażyła?

Że sobie swym posagiem pana nie kupiła?

Że przegróżek i cudzych fuków nie słuchała?

Że boleści w rodzeniu dziatek nie uznała?

Ani umie powiedzieć, czego jej troskliwa

Matka doszła : co z więtszym utrapieniem bywa,

Czy je rodzić, czy je grześć? - Takieć pospolicie

Przysmaki wasze, czym wy sobie świat słodzicie! -

W niebie szczere rozkoszy, a do tego wieczne,

Od wszelakiej przekazy wolne i bezpieczne;

Tu troski nie panują, tu pracej nie znają,

Tu nieszczęście, tu miejsca przygody nie mają,

Tu choroby nie najdzie, tu nie masz starości,

Tu śmierć, łzami karmiona, nie ma już wolności.

Żyjem wiek nieprzeżyty, wiecznej używamy

Dobrej myśli, przyczyny wszytkich rzeczy znamy.

Słońce nam zawżdy świeci, dzień nigdy nie schodzi

Ani za sobą nocy niewidomej wodzi.

Twórcę wszech rzeczy widziem w jego majestacie,

Czego wy, w ciele będąc, prózno upatrzacie.

Tu w czas obróć swe myśli, a chowaj się na te

Nieodmienne, synu mój, rozkoszy bogate!

Doznałeś, co świat umie i jego kochanie;

Lepiej na czym ważniejszym zasadź swe staranie!

Dziewka twoja dobry los, możesz wierzyć, wzięła,

A własnie w swoich rzeczach sobie tak poczęta,

Jako gdy kto, na morze nowo się puściwszy,

A tam niebezpieczeństwo wielkie obaczywszy,

Woli nazad do brzegu. Drudzy, co podali

Żagle wiatrom, na ślepe skały powpadali;

Ten mrozem zwyciężcny, ten od głodu zginął,

Rzadki, co by do brzegu na desce przypłynął.

Śmierci zniknąć nie mogła, by też dobrze była

Onę dawną Sybillę wiekiem swym przeżyła.

To, co mimo być potym, uprzedzić wolała;

Tymże mniej tego świata niewczasów doznała,

Drugie po swych namilszych rodzicach zostają

I ciężkiego siroctwa, nędzne, doznawają.

Wypchną drugą za męża leda jako z domu,

A majętność zostanie, sam to Bóg wie komu.

Biorą drugie i gwałtem, a biorą i swoi,

Ale i w hordach część się wielka ich zostoi,

Gdzie w niewoli pogańskiej i w służbie sromotnej

Łzy swe piją czekając śmierci wszytkokrotnej.

Tego twej wdzięcznej dziewce bać się już nie trzeba,

Która w swych młodych leciech wzięta jest do nieba

Żadnych frasunków tego świata nie doznawszy

Ani grzechem dusze swej drogiej pomazawszy.

Jej tedy rzeczy, synu - nie masz wątpliwości -

Dobrze poszły, ani stąd używaj żałości! -

Swoje szkody tak szacuj i omyłki swoje,

Abyś nie przepamiętał, że baczenie twoje

I stateczność jest droższa! W tę bądź przedsię panem,

Jako się kolwiek czujesz w pociechy obranem.

Człowiek urodziwszy się zasiadł w prawie takim,

Że ma być jako celem przygodom wszelakim;

Z tego trudno się zdzierać! Pocznimy, co chcemy,

Jesli po dobrej woli nie pójdziem, musiemy,

A co wszystkich jednako ciśnie, nie wiem, czemu

Tobie ma być, synu mój, naciężej jednemu.

Śmiertelna jako i ty twoja dziewka była;

Póki jej zamierzony kres był, póty żyła.

Krótko wprawdzie! ale w tym człowiek nic nie włada,

A wyrzec też, co lepiej, niełacno przypada.

Skryte są Pańskie sądy; co się Jemu zdało,

Nalepiej, żeby się też i nam podobało.

Łzy w tej mierze niepłatne; gdy raz dusza ciała

Odbieży, prózno czekać, by się wrócić miała.

Ale człowiek nie zda się praw szczęściu w tej mierze,

Że szkody pospolicie tylko przed się bierze,

A tego baczyć nie chce ani mieć w pamięci,

Co mu też czasem padnie wedle jego chęci.

Tać jest władza Fortuny, mój namilszy synie,

Że nie tak uskarżać się, kiedy nam co zginie,

Jako dziękować trzeba, że wżdam co zostało,

Bo to wszytko nieszczęście w ręku swoich miało.

A tak i ty, folgując prawu powszechnemu,

Zagródź drogę do serca upadkowi swemu

A w to patrzaj, co uszło ręki złej przygody;

Zyskiem człowiek zwać musi, w czym nie popadł szkody

Na koniec, w co się on koszt i ona utrata,

W co się praca i twoje obróciły lata,

Któreś ty niemal wszytkie strawił nad księgami,

Mało się bawiąc świata tego zabawami?

Teraz by owoc zbierać swojego szczepienia

I ratować w zachwianiu mdłego przyrodzenia!

Cieszyłeś przedtym insze w takiejże przygodzie:

I będziesz w cudzej czulszy niżli w swojej szkodzie?

Teraz, mistrzu, sam się lecz! Czas doktór każdemu,

Ale kto pospolitym torem gardzi, temu

Tak póznego lekarstwa czekać nie przystoi!

Rozumem ma uprzedzić, co insze czas goi.

A czas co ma za fortel? Dawniejsze świeżymi

Przypadkami wybija, czasem weselszymi,

Czasem też z tejże miary; co człowiek z baczeniem

Pierwej, niż przyjdzie, widzi i takim myśleniem

Przeszłych rzeczy nie wściąga, przyszłych upatruje

I serce na oboję fortunę gotuje.

Tego się, synu, trzymaj, a ludzkie przygody

Ludzkie noś; jeden jest Pan smutku i nagrody."

Tu zniknęła. - Jam się też ocknął. - Aczciem prawie

Niepewien, jeslim przez sen słuchał czy na jawie.

\newpage
\subsection{Obrazek  w tekście} 

Z tego trudno się zdzierać! Pocznimy, co chcemy,
Jesli po dobrej woli nie pójdziem, musiemy,
A co wszystkich jednako ciśnie, nie wiem, czemu
Tobie ma być, synu mój, naciężej jednemu.
Śmiertelna jako i ty twoja dziewka była;
Póki jej zamierzony kres był, póty żyła.
Krótko wprawdzie! ale w tym człowiek nic nie włada,
A wyrzec też, co lepiej, niełacno przypada.
Skryte są Pańskie sądy; co się Jemu zdało,
Nalepiej, żeby się też i nam podobało.
Łzy w tej mierze niepłatne; gdy raz dusza ciała
Odbieży, prózno czekać, by się wrócić miała.
Ale człowiek nie zda się praw szczęściu w tej mierze,
Że szkody pospolicie tylko przed się bierze,
A tego baczyć nie chce ani mieć w pamięci,
Co mu też czasem padnie wedle jego chęci.
Tać jest władza Fortuny, mój namilszy synie, \label{o1}
\begin{wrapfigure}{l}{0.25\textwidth}
    \centering
    \includegraphics[width=0.25\textwidth]{malloy.jpg}
\end{wrapfigure}
Że nie tak uskarżać się, kiedy nam co zginie,
Jako dziękować trzeba, że wżdam co zostało,
Bo to wszytko nieszczęście w ręku swoich miało.
A tak i ty, folgując prawu powszechnemu,
Zagródź drogę do serca upadkowi swemu
A w to patrzaj, co uszło ręki złej przygody;
Zyskiem człowiek zwać musi, w czym nie popadł szkody
Na koniec, w co się on koszt i ona utrata,
W co się praca i twoje obróciły lata,
Któreś ty niemal wszytkie strawił nad księgami,
Mało się bawiąc świata tego zabawami?
Teraz by owoc zbierać swojego szczepienia
I ratować w zachwianiu mdłego przyrodzenia!
Cieszyłeś przedtym insze w takiejże przygodzie:
I będziesz w cudzej czulszy niżli w swojej szkodzie?
Teraz, mistrzu, sam się lecz! Czas doktór każdemu,
Ale kto pospolitym torem gardzi, temu
Tak póznego lekarstwa czekać nie przystoi!
Rozumem ma uprzedzić, co insze czas goi.
A czas co ma za fortel? Dawniejsze świeżymi
Przypadkami wybija, czasem weselszymi,
Czasem też z tejże miary; co człowiek z baczeniem
Pierwej, niż przyjdzie, widzi i takim myśleniem
Przeszłych rzeczy nie wściąga, przyszłych upatruje
I serce na oboję fortunę gotuje.
Tego się, synu, trzymaj, a ludzkie przygody
Ludzkie noś; jeden jest Pan smutku i nagrody."
Tu zniknęła. - Jam się też ocknął. - Aczciem prawie
Niepewien, jeslim przez sen słuchał czy na jawie.

\newpage
\section{Jakieś smieszne krzaczki}

\subsection{Falki} 

$$\int_{a}^b f(x)dx$$

I takie nawet:

$$\int_{a}^b\int_{c}^d\int_{e}^f\int_{g}^h\int_{i}^j f(x,y,z,k,l)dxdydzdkdl$$

\subsection{I w drugą stronę} 

$$\frac{\partial^2 f}{\partial x^2}$$  $$f^{(k)}(x)$$
A między literkami też się $\sum^k_{i=1}\sum^l_{j=1}\,q_i q_j$ da chyba.

\subsection{A takie też można} 

$$\sum_{i=1}^n i^2 = \frac{n(n+1)(2n+1)}{6}$$

\section{Podsumowanie}

Obrazek 1 - strona \pageref{o0}.
Obrazek w tekście - strona \pageref{o1}.
Tabela -strona \pageref{tab}.

\newpage
\section{Bibliografia}\label{bb}
\subsection{Źródła}
M. Miotk "Nie za krótkie wprowadzenie do LaTeXa"
https://literat.ug.edu.pl/

\end{document}